\documentclass[10pt, letterpaper]{article}

% Packages:
\usepackage[
    ignoreheadfoot, % set margins without considering header and footer
    top=2 cm, % seperation between body and page edge from the top
    bottom=2 cm, % seperation between body and page edge from the bottom
    left=2 cm, % seperation between body and page edge from the left
    right=2 cm, % seperation between body and page edge from the right
    footskip=1.0 cm, % seperation between body and footer
    % showframe % for debugging
]{geometry} % for adjusting page geometry
\usepackage{titlesec} % for customizing section titles
\usepackage{tabularx} % for making tables with fixed width columns
\usepackage{array} % tabularx requires this
\usepackage[dvipsnames]{xcolor} % for coloring text
\definecolor{primaryColor}{RGB}{0, 0, 0} % define primary color
\usepackage{enumitem} % for customizing lists
\usepackage{fontawesome5} % for using icons
\usepackage{amsmath} % for math
\usepackage[
    pdftitle={John Doe's CV},
    pdfauthor={John Doe},
    pdfcreator={LaTeX with RenderCV},
    colorlinks=true,
    urlcolor=primaryColor
]{hyperref} % for links, metadata and bookmarks
\usepackage[pscoord]{eso-pic} % for floating text on the page
\usepackage{calc} % for calculating lengths
\usepackage{bookmark} % for bookmarks
\usepackage{lastpage} % for getting the total number of pages
\usepackage{changepage} % for one column entries (adjustwidth environment)
\usepackage{paracol} % for two and three column entries
\usepackage{ifthen} % for conditional statements
\usepackage{needspace} % for avoiding page brake right after the section title
\usepackage{iftex} % check if engine is pdflatex, xetex or luatex

% Ensure that generate pdf is machine readable/ATS parsable:
\ifPDFTeX
    \input{glyphtounicode}
    \pdfgentounicode=1
    \usepackage[T1]{fontenc}
    \usepackage[utf8]{inputenc}
    \usepackage{lmodern}
\fi

\usepackage{charter}

% Some settings:
\raggedright
\AtBeginEnvironment{adjustwidth}{\partopsep0pt} % remove space before adjustwidth environment
\pagestyle{empty} % no header or footer
\setcounter{secnumdepth}{0} % no section numbering
\setlength{\parindent}{0pt} % no indentation
\setlength{\topskip}{0pt} % no top skip
\setlength{\columnsep}{0.15cm} % set column seperation
\pagenumbering{gobble} % no page numbering

\titleformat{\section}{\needspace{4\baselineskip}\bfseries\large}{}{0pt}{}[\vspace{1pt}\titlerule]

\titlespacing{\section}{
    % left space:
    -1pt
}{
    % top space:
    0.3 cm
}{
    % bottom space:
    0.2 cm
} % section title spacing

\renewcommand\labelitemi{$\vcenter{\hbox{\small$\bullet$}}$} % custom bullet points
\newenvironment{highlights}{
    \begin{itemize}[
        topsep=0.10 cm,
        parsep=0.10 cm,
        partopsep=0pt,
        itemsep=0pt,
        leftmargin=0 cm + 10pt
    ]
}{
    \end{itemize}
} % new environment for highlights


\newenvironment{highlightsforbulletentries}{
    \begin{itemize}[
        topsep=0.10 cm,
        parsep=0.10 cm,
        partopsep=0pt,
        itemsep=0pt,
        leftmargin=10pt
    ]
}{
    \end{itemize}
} % new environment for highlights for bullet entries

\newenvironment{onecolentry}{
    \begin{adjustwidth}{
        0 cm + 0.00001 cm
    }{
        0 cm + 0.00001 cm
    }
}{
    \end{adjustwidth}
} % new environment for one column entries

\newenvironment{twocolentry}[2][]{
    \onecolentry
    \def\secondColumn{#2}
    \setcolumnwidth{\fill, 4.5 cm}
    \begin{paracol}{2}
}{
    \switchcolumn \raggedleft \secondColumn
    \end{paracol}
    \endonecolentry
} % new environment for two column entries

\newenvironment{threecolentry}[3][]{
    \onecolentry
    \def\thirdColumn{#3}
    \setcolumnwidth{, \fill, 4.5 cm}
    \begin{paracol}{3}
    {\raggedright #2} \switchcolumn
}{
    \switchcolumn \raggedleft \thirdColumn
    \end{paracol}
    \endonecolentry
} % new environment for three column entries

\newenvironment{header}{
    \setlength{\topsep}{0pt}\par\kern\topsep\centering\linespread{1.5}
}{
    \par\kern\topsep
} % new environment for the header

\newcommand{\placelastupdatedtext}{% \placetextbox{<horizontal pos>}{<vertical pos>}{<stuff>}
  \AddToShipoutPictureFG*{% Add <stuff> to current page foreground
    \put(
        \LenToUnit{\paperwidth-2 cm-0 cm+0.05cm},
        \LenToUnit{\paperheight-1.0 cm}
    ){\vtop{{\null}\makebox[0pt][c]{
        \small\color{gray}\textit{Last updated in September 2024}\hspace{\widthof{Last updated in September 2024}}
    }}}%
  }%
}%

% save the original href command in a new command:
\let\hrefWithoutArrow\href

% new command for external links:


\begin{document}
    \newcommand{\AND}{\unskip
        \cleaders\copy\ANDbox\hskip\wd\ANDbox
        \ignorespaces
    }
    \newsavebox\ANDbox
    \sbox\ANDbox{$|$}

    \begin{header}
        \fontsize{25 pt}{25 pt}\selectfont Zhanchao Yang

        \vspace{4 pt}

        \normalsize
        \kern 5.0 pt%
        \mbox{\hrefWithoutArrow{mailto:zhanchao@upenn.edu}{zhanchao@upenn.edu}}%
        \kern 5.0 pt%
        \AND%
        \kern 5.0 pt%
        \mbox{\hrefWithoutArrow{tel:6076447982}{607 644 7982}}%
        \kern 5.0 pt%
        \AND%
        \kern 5.0 pt%
        \mbox{\hrefWithoutArrow{https://bit.ly/zhanchao  }{bit.ly/zhanchao  }}%
        \kern 5.0 pt%
        \AND%
        \kern 5.0 pt%
        \mbox{\hrefWithoutArrow{https://zhanchaoyang.weebly.com/ }{zhanchaoyang.weebly.com/ }}%
        \kern 5.0 pt%
        \AND%
        \mbox{510 N Broad St, Philadelphia, PA, 19130}%
        \kern 5.0 pt%
        \AND%
        \kern 5.0 pt%
        \mbox{\hrefWithoutArrow{https://github.com/zyang91}{github.com/zyang91}}%
    \end{header}

    \vspace{5 pt - 0.3 cm}


    \section{EDUCATION}




        \begin{twocolentry}{
            August 2024 – May 2026
        }
            \textbf{University of Pennsylvania}, Weitzman School of Design\end{twocolentry}

        \vspace{0.10 cm}
        \begin{onecolentry}
            \begin{highlights}
                \item Master of City and Regional Planning | Master of Urban Spatial Analytics
                \item  GPA: 3.93/4.0
            \end{highlights}
        \end{onecolentry}

        \vspace{0.2 cm}

        \begin{twocolentry}{
            August 2020 – May 2024
        }
            \textbf{Binghamton University}, State University of New York\end{twocolentry}

        \vspace{0.10 cm}
        \begin{onecolentry}
            \begin{highlights}
                \item \textbf{Bachelor of Arts} in Geography | Minors in Environmental Study \& Global Studies
                \item  Summa Cum Laude | Highest departmental honors in Geography
                \item  Overall GPA: 3.88/4.0 | Dean’s list (2020-2024)
            \end{highlights}
        \end{onecolentry}



    \section{\textbf{EXTRACURRICULAR EXPERIENCE }
}




        \begin{twocolentry}{
            10/2024 – Present
        }
\textbf{Graduate Research Assistant}, University of Pennsylvania-- Philadelphia, PA\end{twocolentry}

        \vspace{0.10 cm}
        \begin{onecolentry}
            \begin{highlights}
                \item Utilized R to analyze the Fatality Analysis Reporting System (FARS) datasets from the past 10 years and aggregated the data specifically for children by Metropolitan Statistical Areas (MSAs).
                \item Presented at weekly meetings and collaborated with other research team members and professors to provide the latest updates and share insights.
                \item Conducted a thorough literature review on surveying and interviewing methods related to transportation related child fatalities and advanced transportation technologies.
            \end{highlights}
        \end{onecolentry}


        \vspace{0.2 cm}

        \begin{twocolentry}{
            01/2024 – 06/2024
        }
            \textbf{\textbf{Member of EML Lecturer Hiring Committee}}-- Binghamton University \end{twocolentry}

        \vspace{0.10 cm}
        \begin{onecolentry}
            \begin{highlights}
                \item Collaborated with five other English Language professors and practitioners to evaluate potential candidates, ensuring that the new lecturers would meet the needs of international students.
                \item Used my background in English learning and cultural adaptation to work with other committee members on crafting suitable interview questions.
                \item Organized and led Students’ Q\&A sessions during the candidate’s on-campus visit, and collected participants’ feedback and comments after each session.
            \end{highlights}
        \end{onecolentry}

        \vspace{0.2 cm}

        \begin{twocolentry}{
            09/2023 – 05/2024
        }
            \textbf{\textbf{\textbf{Undergraduate Research Assistant}}}, Near Earth Imaging Lab, Binghamton   \end{twocolentry}
        \vspace{0.10 cm}
        \begin{onecolentry}
            \begin{highlights}
                \item Employed Python and R to analyze satellite imagery and drone flight images, aiding in developing an advanced near-ground imaging processing workflow.
                \item Presented in weekly lab meetings and collaborated with other team members and professors to discuss project progress, share insights, and develop innovative solutions.
                \item Coordinated with colleagues on different aspects of the project, promoting a collaborative atmosphere to accomplish research goals and enhance team efficiency.
            \end{highlights}
        \end{onecolentry}




    \section{AWARDS}

        \begin{onecolentry}
                    \textbf{\textbf{Binghamton University President Award for Student Excellence-Honorable mentioned (2024)}}
        \end{onecolentry}

                \begin{onecolentry}
                    \textbf{\textbf{\textbf{SUNY Chancellor Award for Student Excellence (2024)}
}}
        \end{onecolentry}
                \vspace{0.10 cm}
        \begin{onecolentry}
            \begin{highlights}
                \item The highest honor that can be achieved upon a student by the State University of New York.
                \item Nominated By Binghamton University President Harvey Stenger
            \end{highlights}
        \end{onecolentry}








    \section{SKILLS}




        \begin{onecolentry}
            \textbf{\textbf{Programming languages: }}R (Intermediate), Python (Advanced)\textbf{ }
        \end{onecolentry}

        \vspace{0.1 cm}

        \begin{onecolentry}
            \textbf{\textbf{Data Analytics Software: }}ArcGIS (ESRI) Suite(Advanced), PIX4D(Advanced), QGIS(Intermediate)

        \end{onecolentry}

         \vspace{0.1 cm}

        \begin{onecolentry}
                    \textbf{\textbf{\textbf{Visual and Modeling: }}}Microsoft Office Suite (Advanced), Adobe Illustrator (Intermediate)

        \end{onecolentry}

         \vspace{0.1 cm}

        \begin{onecolentry}
                    \textbf{\textbf{\textbf{\textbf{Language: }}}}Chinese (Native), English (Full professional proficiency)

        \end{onecolentry}





\end{document}
